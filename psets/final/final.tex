%%%%%%%%%%%%%%%%%%%%%%%%%%%%%% Preamble
\documentclass{article}
\usepackage{amsmath,amssymb,amsthm,fullpage}
\usepackage[a4paper,bindingoffset=0in,left=1in,right=1in,top=1in,
bottom=1in,footskip=0in]{geometry}
\newtheorem*{prop}{Proposition}
%\newcounter{Examplecount}
%\setcounter{Examplecount}{0}
\newenvironment{discussion}{\noindent Discussion.}{}
\pagenumbering{gobble}
\begin{document}

%%%%%%%%%%%%%%%%%%%%%%%%%%%%%% Heading
%\title{CS21 Midterm}
%\author{Matt Lim}
%\date{February 12, 2014}
%\maketitle



%%%%%%%%%%%%%%%%%%%%%%%%%%%%%% Problem 1
\section*{Problem 1, CS21 Final, Matt Lim}
\begin{description}
    \item[(a)]
        We will prove that $L$ is in PSPACE. To do this, we will consider the
        algorithm that decides this problem. For the odd numbered moves (player
        1), we can do the following. For both moves $S_i$ and $T_i$, we can
        recursively solve the game for the situation where player 2 gets to
        choose from $S_{i+1}$ or $T_{i+1}$ (with the chosen subset
        $S_i$ or $T_i$ having been deleted from $G$, along with their
        incident edges). If at least one of the moves of player 1
        returns a ``win'' for player 1, return ``yes''. Else, return ``no''. For the
        even numbered moves (player 2), we can do the following. For both
        $S_j$ and $T_j$, we can recursively solve the game for the situation
        where player 1 gets to choose from $S_{j+1}$ or $T_{j+1}$ ( with the
        chosen subset $S_j$ or $T_j$ having been deleted from $G$, along with their
        incident edges).  If at least
        one of the moves from player 2 doesn't return a ``win'' for player 1, return ``no''.
        Else, return ``yes''. The base case is just determining whether or not
        a graph contains a $k$-clique. This can clearly be done in polynomial
        space by simply looking at all combinations of $k$ or more vertices
        one at a time and determining whether any are cliques. The recursion depth
        is also polynomial, as is the bits of state at each level (we actually lose
        a pair of subsets at each level of recursion).  So we have that $L$ is in PSPACE.
    \item[(b)]
        We will prove that $L$ is PSPACE-hard. To do this we will reduce QSAT
        to $L$. Our reduction function $f$ will be as follows. For each
        clause in $\varphi$, we will generate a triple of nodes. If a clause
        has less than two nodes, then our triple will simply contain duplicates
        of the literals in that clause. We will then
        put all possible edges between different triples. We will let the
        subset of vertices $x_i$ be $S_i$, and the subset of vertices
        $\neg x_i$ be $T_i$. Then we will let $k=m$, where $m$ is the number
        of clauses in the instance of QSAT we reduce from. Observing this
        reduction function, it is clearly polynomial. Let's go through each
        step to make sure. For each clause, we generate a triple of nodes,
        a polynomial process. Then we put all possible edges between the
        different triples, also a polynomial process. Finally, we map each set
        of vertices $x_i$ to a $S_i$ and each set of vertices $\neg x_i$ to a
        $T_i$. All these processes are polynomial, so we have that our reduction
        function $f$ is polynomial.

        Now we will show that this reduction function $f$ maps yes to yes and no
        to no. First we will show yes maps to yes. So, given that $\varphi$
        is in QSAT, we want to show that the
        $(G,k, (S_1, T_1), \dots, (S_n, T_n))$ it maps to is an instance of
        $L$. So, assume that $\varphi$ is a yes instance of QSAT. This means
        $\exists x_i \forall x_2 \exists x_3 \forall x_4 \exists x_5 \dots
        \forall x_n$ such that $\varphi$ is satisfied. So basically,
        ``player 1'' can satisfy $\varphi$ no matter what ``player 2'' does. Now
        we will observe how this translates to our graph. For each literal
        value that we pick, we delete a corresponding subset of vertices.
        Consider what happens when we pick the opposite set of vertices; that
        is, if $x_i$ is true in $\varphi$, we will pick the subset of $\neg
        x_i$s to be deleted. Since there must be at least one true literal in
        each clause (since $\varphi$ is satisfiable), at the end of the game,
        there will be at least $m$ vertices, one from each triple. And each of
        these $m$ vertices will be connected to each other, since we put all possible edges
        between different triples. So the graph will have a clique of size $m$
        no matter what. And since we let $k=m$, we have that $(G,k, (S_1, T_1),
        \dots, (S_n, T_n))$ is a yes instance of $L$. Thus yes maps to yes.

        Now we will show that this reduction function $f$ maps no to no. To do
        this, it suffices to show that if $(G,k, (S_1, T_1), \dots, (S_n, T_n))$
        is in $L$, then $\varphi$ is in QSAT. So, assume we have that
        $(G,k, (S_1, T_1), \dots, (S_n, T_n))$ is in $L$. This means that player
        1 can win no matter what player 2 does. So player 1 can make it so that
        at the end of the game, there is always a $k$-clique. But there can only
        be a $k$-clique if at least one vertex from every triple of vertices
        remains, because the triples of vertices aren't connected within
        themselves. And we have that, of the remaining vertices, there cannot
        be any contradictory vertices (because each turn we either delete
        all $x_i$'s or all $\neg x_i$'s). This means that the vertices at the
        end of the game represent a satisfying assignment for $\varphi$.
        So in total, we have that player 1 can always make a $k$ clique appear
        at the end of the game, which in turn means that player 1 can always
        satisfy $\varphi$. Thus no maps to no.

        We proved above that our reduction function $f$ is polynomial. So
        finally, we can conclude that $L$ is PSPACE-hard.
\end{description}
\newpage

%%%%%%%%%%%%%%%%%%%%%%%%%%%%%% Problem 2
\section*{Problem 2, CS21 Final, Matt Lim}
We will prove that $L$ is context-free but not regular. First we will prove that
$L$ is not regular. To do this, we can use the fact that regular languages
are closed under complement. This means that if the complement of $L$ is not
regular, then $L$ itself is not regular. We have that the complement of $L$,
$\overline{L}$, is all strings with an equal number of $a$'s and $b$'s (and any number of
$c$'s). We will prove that $\overline{L}$ is not regular by using the pumping
lemma.

Assume that $\overline{L}$ is regular. Then there exists a pumping length $p$. We will
choose the string $w = a^pb^pc \in \overline{L}$. We can write this string as
$w = xyz$, with $|y| > 0$ and $|xy| \leq p$. Since $|xy| \leq p$, then $y$ must be contained
in the first $p$ $a$'s (i.e. $y$ can only be in the $a$'s). So, choose any $y$
in the first $p$ $a$'s. Pumping on $y$ gives us an unequal number of $a$'s and
$b$'s. So pumping on $y$ gives us a string that is not in the language. But
this is a contradiction. Thus we have that $\overline{L}$ is not regular, and
thus that $L$ is not regular.

Now we will prove that $L$ is context free using the following statement: $L$
is recognized by a NPDA implies $L$ is described by a CFG. We will construct
an NPDA that recognizes $L$, thus proving $L$ is context free.
 Look at the NPDA constructed on the following page. Let $k \neq a,b,c$ and let
 $m$ be any symbol. As we can see, $L$ is recognized by our NPDA. Thus, $L$ is
 context free.
\newpage

%%%%%%%%%%%%%%%%%%%%%%%%%%%%%% Problem 3
\section*{Problem 3, CS21 Final, Matt Lim}
To show that $L$ is neither R.E. nor co-R.E., we will show that $EQ_{TM}$, which
is neither R.E. nor co-R.E. (as proved in the book and in lecture) is
reducible to $L$. The reducing function $f$ works as follows. On an input
$\langle M_1, M_2 \rangle$, we will map this to output $\langle M_2, M_1, M_2
\rangle$, where $M_1$ and $M_2$ are the same as they are in the input. So,
if $\langle M_1, M_2 \rangle$ is a yes instance of $EQ_{TM}$, then
$L(M_1) = L(M_2) \implies L(M_2) \subseteq L(M_1) \subseteq L(M_2)$, which means
that $\langle M_2, M_1, M_2 \rangle$ is a yes instance of $L$. So yes maps to
yes. If $\langle M_1, M_2 \rangle$ is a no instance of $EQ_{TM}$, then we
have three possible cases. The first is the case in which $L(M_1)$ contains a
string (or strings) not in $L(M_2)$. Then we have that $L(M_1) \nsubseteq
L(M_2)$, and thus that no maps to no. The second is the case in which $L(M_2)$
contains a string (or strings) not in $L(M_1)$. Then we have that $L(M_2)
\nsubseteq L(M_1)$, and thus that no maps to no. The third is the case in which
$L(M_1)$ contains a string (or strings) not in $L(M_2)$ and $L(M_2)$ contains
a string (or strings) not in $L(M_1)$. So we have that neither of them are
subsets of each other (or equal to each other) and thus that no maps to no. So
we have reduced $EQ_{TM}$ to $L$ and thus shown that $L$ is neither R.E. nor co-R.E..
\newpage

%%%%%%%%%%%%%%%%%%%%%%%%%%%%%% Problem 3
\section*{Problem 4, CS21 Final, Matt Lim}
\begin{description}
    \item[(a)]
        Assume that the language $L$ is regular. Now we will show that
        the language
        \[ L_{-y} = \{xz : x \in \Sigma^* \text{ and } z \in \Sigma^* \text{ and
        } xyz \in L\} \]
        which consists of all strings in $L$ with the string $y$ deleted from
        them is also regular. To do this, we will use the theorem that a
        language $L$ is recognized by a FA if and only if $L$ is described by
        a regular language. So, from this theorem, we have that $L$ is
        recognized by some FA. Now we must only shows that $L_{-y}$ is also
        recognized by some FA (or some NFA, since FAs and NFAs are equivalent).
        To do this, we will consider the FA that recognizes $L$. Let us call
        it $A$. We have that, since $A$ recognizes an arbitary $xyz \in L$, there
        is some part of $A$, composed of states and transitions, that reads through
        $x \in \Sigma^*$, some part of $A$ that reads through $y \in \Sigma^*$,
        and some part of $A$ that reads through $z \in \Sigma^*$. Now, consider
        the state that $A$ is in when it reaches the last symbol of $x$. This
        state must have some transition that reads the first symbol of $y$. Now,
        consider what happens if we delete this transition, and add a transition that reads
        the first symbol of $z$. We effectively now have a FA that reads to the
        last symbol of $x$, skips the $y$ string entirely, and transitions from
        the last symbol of $x$ to the first symbol of $z$. In other words, we
        have a FA that recognizes $L_{-y}$. So we can conclude that $L_{-y}$ is
        regular.
    \item[(b)]
        Assume that $L$ is R.E.. Now we will show that $L_{-y}$ is also R.E..
        We have that a language is R.E. if it is recognized by some TM. We will
        let $M_L$ be the machine that recognizes $L$. Now we must design some
        machine $M$ that recognizes $L_{-y}$. We will fix some enumeration
        $y_1, y_2, \dots$
        of $\Sigma^*$ in lexicographic order. Our machine $M$ for $L_{-y}$ does
        the following for $i = 1, 2, 3, \dots$: simulates machine
        $M_L$ on the inputs $xy_1z, xy_2z, \dots, xy_iz$ for $i$ steps, for
        each way we can split $xz$ into $x \in \Sigma^*$ and $z \in \Sigma^*$. If
        during any of the simulations $M_L$ accepts some $xy_jz$, then $M$
        halts and accepts.

        Suppose $xz$ is in $L_{-y}$. Then there is some $xyz$ (for some
        splitting $x$ and $z$ of $xz$) which is accepted
        by $M_L$ after $n_L$ steps. If $y$ appears in the lexicographic order
        at position $m$, then our machine $M$ will accept when $i = max(m,
        n_L)$. Thus every string $xz \in L_{-y}$ is accepted by $M$.

        Conversely, if $M$ accepts some string $xz$, then it can only have done
        so because for some value $i$, there was some string $y_j$ with
        $j \leq i$ for which $M_L$ accepted $xy_jz$ for some splitting
        $x$ and $z$ of $xz$. Therefore there is a string
        $xy_jz$ in $L$ for some splitting $x$ and $z$ of $xz$, which implies $xz \in
        L_{-y}$ as required.
\end{description}
\newpage

%%%%%%%%%%%%%%%%%%%%%%%%%%%%%% Problem 3
\section*{Problem 5, CS21 Final, Matt Lim (a, b, d, e)}
\begin{description}
    \item[(a)]
        We will show that the language $L$ described in part (a) is NP-complete.
        To do this, we must show that it is in NP and that all NP-problems are
        polynomial time reducible to it. To prove the first part, we can simply
        give a certificate and a polynomial time verifier. The certificate is
        just an independent set of $G$ with size at least $k$. The verifier will
        iterate through the vertices in the set and check that no pair of vertices
        makes up an edge in $G$. It will also iterate through all vertices
        of $G$ and make sure that no vertex has more than 4 edges touching it.
        Both operations are clearly polynomial.

        To prove the second part, we show that INDEPENDENT SET (IS) is polynomial time
        reducible to $L$. The following details the reduction.

        Our reduction function $f$ will map $(G,k)$ (a graph and an integer) to
        $(G',k')$ (another graph and an integer. It will do this in the
        following way. We will let $k' = k$, so we will just copy over $k$.
        To make $G'$, we will do the following. We will iterate through all the
        vertices of graph $G$, and for each vertex that has more than 4 edges,
        we will delete edges from it until it only has 4 edges. This ``trimmed''
        graph will be our $G'$.

        Now we will show that this reduction function $f$ maps yes to yes and
        no to no. First we will show that yes maps to yes. So, assume that
        $(G,k)$ is in IS. That means it has an independent set of size at least
        $k$. Then we have that our $G'$ also has an independent set of size at
        least $k$, since deleting edges from $G$ cannot possibly decrease the
        size of its independent set (by the definition of an independent set).
        And since $k' = k$, and since we trimmed the number of edges so that
        $G'$ has no vertices with more than 4 edges that touch it, we have that
        $(G',k')$ is in $L$.

        Now we will show that this reduction function $f$ maps no to no. To do
        this, it suffices to show that if $(G',k')$ is in $L$, then $(G,k)$ is
        in IS. So, assume $(G',k')$ is in $L$. Then we have that $G'$ has
        an independent set of size at least $k'$. Now consider its corresponding $G$.
        $G'$ is basically the same as $G$, except with some edges trimmed. So
        let us consider what happens when we add those edges back to form $G$.
        At least $k'$ vertices in an independent set of $G'$ are not affected at all,
        since those same vertices in $G$ had no edges between them. And since
        $k' = k$, we have that $G$ has an independent set of size at least $k$.
        So $(G,k)$ is in IS and no maps to no.

        Now we must only prove that our reduction function $f$ is polynomial
        time computable. Our function is clearly polynomial. We simply iterate
        through all the vertices (polynomial), delete a polynomially large
        number of edges from each vertex, then map that trimmed graph and an
        integer $k$ to a new graph and new integer $k'$. So we have that our
        reduction function is polynomial.

        Finally, we can conclude that $L$ is NP-complete.
    \item[(b)]
        We will prove that the language described in part (b) is in P by
        giving a polynomial algorithm for it. Consider the case when $k$ is over
        101. Then we have that $(G,k)$ cannot be in $L$, because since $G$ must
        have a maximum degree of 100, the maximum clique it can have is a
        101-clique. Now consider the case when $k$ is 101 and below. Then we
        can treat $k$ as a fixed number, and solve for the fixed numbers 1-101.
        For each fixed number $m$ (and letting $y$ be the number of vertices in
        $G$) we can try all the $yCm$ ways that $m$ vertices can be chosen in
        $G$. This is $O(y^m)$ time. Then, for each way that $m$ vertices can be
        chosen, we will see if they form a clique. This is clearly polynomial
        because there are a polynomially large number of edges in $G$. Then
        we will record our result for each $m$. So since for each $m$ the
        process is polynomial, then the process as a whole is also polynomial.
        Then at the end, we will iterate
        through all the results until we reach the $k$th one. If that value
        is yes, we accept. If it is no, we reject.
    \item[(d)]
        We will show that the language described in part (d) is in P by
        reducing it to 2-SAT, which is in P.

        We will call the language in part (d) $L$. So, take an arbitrary
        $\phi$ in $L$. Our reduction function $f$ will map this to a 2-CNF
        formula $\varphi$. It will do this by trying all possible ways to put
        750 of the first 1000 clauses along with the rest of them. So, it will
        try all 1000 choose 750 ways to choose 750 of the first 1000 clauses
        (then append the rest of the clauses to the chosen 750)
        and see if they are in 2-SAT (we proved that deciding 2-SAT is polynomial
        time in class). The first time that the chosen set of clauses is in
        2-SAT, $f$ will map those set of clauses to be $\varphi$. If none of
        the chosen set of clauses is in 2-SAT,
        $f$ will just copy over $\phi$ to $\varphi$.

        Now we must show that $f$ maps yes to yes and no to no. First we will
        show that it maps yes to yes. So, assume $\phi$ is a yes instance
        of $L$. Then we have that there is some set of 750 clauses from the
        first 1000 clauses that is completely satisfiable by the same assignment
        that satisfies all the other clauses. By our reduction function, this
        set of satisfiable clauses (the 750 plus all the other ones)
        is $\varphi$. So $\varphi$ is satisfiable
        and thus is in 2-SAT. Thus we have that yes maps to yes.

        Now we will show that this reduction function $f$ maps no to no. So,
        assume $\phi$ is a no instance of $L$. That means that there does
        not exist an assignment that satisfies all the clauses. And since
        in this case, our reduction function $f$ just copies $\phi$ over to
        $\varphi$, then there is no satisfying assignment for $\varphi$. So
        no maps to no.

        Now we must only prove that our reduction function $f$ is polynomial
        time computable. The first process $f$ goes through is trying all
        1000 choose 750 ways of picking 750 from the first 1000 clauses. This
        is constant time in regards to the input length. Then, it simply maps
        a set of clauses from $\phi$ to $\varphi$, a clearly polynomial
        process. Thus we have that the reduction function is polynomial.

        Finally, we can conclude that $L$ is in P.
    \item[(e)]
        We will show that the language $L$ described in part (e) is NP-complete.
        To do this, we must show that it is in NP and that all NP-problems
        are polynomial time reducible to it. To prove the first part, we can
        simply give a certificate and a polynomial time verifier.
        If $\phi$ is a 2-CNF formula, the
        certificate is simply an assignment that satisfies all the first
        1000 clauses of $\phi$ and at least 3/4 of the other clauses. The
        verifier will check that this assignment indeed does satisfy the
        desired clauses. It can do this by simply iterating through the first
        1000 clauses and seeing if they are all satisfied, then iterating
        through the rest, keeping a counter of how many total other clauses
        there are and a counter of how many of those other clauses are
        satisfied. It will then check that at least 3/4 of those other clauses
        were satisfied by the assignment. The verifier will also make sure
        that the 2-CNF formula has at least 1000 clauses, and that it has
        at most 2 literals per clause. To prove the second part, we show that
        3-SAT is polynomial time reducible to $L$. The following details the
        reduction.

        Our reduction function $f$ will map a 3-CNF formula $\varphi$ to a 2-CNF
        formula $\phi$. It will do this in the following way. First,
        we will make some arbitrary clause $(q \vee q)$ (where $q$ is
        unique with respect to the rest of the clauses not in the first 1000)
        repeat 1000 times. Next, for each clause in
        $\varphi$, we will construct 12 new clauses with 2 literals. Let us consider
        a single clause $(x, y, z) \in \varphi$. We will map this clause to the
        10 new 2-literal clauses shown below:
        \[ (x \vee x), (y \vee y), (z \vee z), (w \vee w), \]
        \[ (\neg x \vee \neg y), (\neg y \vee \neg z), (\neg z \vee \neg x), \]
        \[ (x \vee \neg w), (y \vee \neg w), (z \vee \neg w) \]
        \[ (a \vee a), (a \vee a) \]
        Consider the truth table for these 12 clauses, as shown below, letting $1$
        represent true, $0$ represent false, and max be the max number of clauses
        that are simultaneously satisfiable (depending on what $w$ is assigned
        and on what $a$ is assigned):
        \begin{center}
            \begin{tabular}{ | l | l | c | r | }
                \hline
                x & y & z & max \\ \hline
                1 & 1 & 1 & 9 \\ \hline
                1 & 0 & 0 & 9 \\ \hline
                0 & 1 & 0 & 9 \\ \hline
                0 & 0 & 1 & 9 \\ \hline
                1 & 1 & 0 & 9 \\ \hline
                0 & 1 & 1 & 9 \\ \hline
                1 & 0 & 1 & 9 \\ \hline
                0 & 0 & 0 & 8 \\ \hline
            \end{tabular}
        \end{center}
        Note that the maximum number of clauses that can be satisfied here is 9. Also
        note that this happens exactly when $(x \vee y \vee z)$ is true. So, this
        is our reduction function.

        Now we will show that this reduction function $f$ maps yes to yes and no
        to no. First we will show yes maps to yes. So, given that $\varphi$
        has a satisfying assignment, we want to show that $\phi$ is in $L$.
        So, assume $\varphi$ has a satisfying assignment. The first 1000
        literals in $\phi$ are all trivially satisfiable by setting $q$ to be
        true. And they are satisfiable no matter what the other clauses are,
        since $q$ only appears in the first 1000 clauses. Now, for the other
        clauses. We have that for each clause in $\varphi$, $(x \vee y \vee z)$
        is true. This means that for each of the 12 clauses we generate to put
        in $\phi$ from a single clause in $\varphi$, 9/12 of them can be made
        true. And this is true for every set of 12 clauses generated, which makes up all
        the other clauses in $\phi$. So we have that 9/12 of all the other
        clauses are satisfiable, which means that at least 3/4 of the other
        clauses are satisfiable. So we have that there is an assignment that
        satisfies all the first 1000 clauses (trivially) and at least 3/4 of the
        other clauses.

        Now we will show that this reduction function $f$ maps no to no. So,
        consider the other clauses (the ones after the first 1000). We have that
        $\varphi$ is not satisfiable. This means that $(x \vee y \vee z)$
        cannot be made true for every clause in $\varphi$. This means that, for
        at least one set of 12 clauses in $\phi$, less than 9 clauses can be
        made true. But since the maximum number of clauses that can be made true
        for every set of 12 is only 9, this means that less than 9/12 of all the
        other clauses can be made true. So we have that $\phi$ is not in $L$ and
        thus that no maps to no.

        Now we must only prove that our reduction function $f$ is polynomial time
        computable. Our function is clearly polynomial, because for each clause
        in $\varphi$, it simply generates 12 clauses to put in $\phi$. So we have
        that our reduction function is polynomial.

        Finally, we can conclude that $L$ is NP-complete.
\end{description}
\newpage
%%%%%%%%%%%%%%%%%%%%%%%%%%%%%% Problem 3
\end{document}

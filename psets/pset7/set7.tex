%%%%%%%%%%%%%%%%%%%%%%%%%%%%%% Preamble
\documentclass{article}
\usepackage{amsmath,amssymb,amsthm,fullpage}
\usepackage[a4paper,bindingoffset=0in,left=1in,right=1in,top=1in,
bottom=1in,footskip=0in]{geometry}
\newtheorem*{prop}{Proposition}
%\newcounter{Examplecount}
%\setcounter{Examplecount}{0}
\newenvironment{discussion}{\noindent Discussion.}{}
\pagenumbering{gobble}
\begin{document}

%%%%%%%%%%%%%%%%%%%%%%%%%%%%%% Heading
%\title{CS21 Midterm}
%\author{Matt Lim}
%\date{February 12, 2014}
%\maketitle



%%%%%%%%%%%%%%%%%%%%%%%%%%%%%% Problem 1
\section*{Problem 1, CS21 Set 7, Matt Lim}
To show that the language
\[ L = \{y: g(f^{-1}(y)) = 1\} \]
is in NP $\cap$ coNP, we will show that it is both in NP and coNP.
To prove this, we can simply give a certificate and polynomial time verifier
for both NP and coNP.

To show $L$ is in NP, the certificate is $x$ such that
$g(x) = 1$. The verifier will check that $g(x) = 1$, a process which is
polynomial in time since $g$ is a polynomial-time computable predicate.

To show $L$ is in coNP, we must simply show that $co-L$ is in NP. We have that,
since $g(x)$ is a bit,
\[ co-L = \{y: g(f^{-1}(y)) = 0\} \]
To show $co-L$ is in NP, the certificate is $x$ such that
$g(x) = 0$. The verifier will check that $g(x) = 0$, a process which is
polynomial in time since $g$ is a polynomial-time computable predicate.

So we have shown that $L$ is in NP and coNP. So $L$ is in NP $\cap$ coNP.
\newpage

%%%%%%%%%%%%%%%%%%%%%%%%%%%%%% Problem 2
\section*{Problem 2, CS21 Set 7, Matt Lim}
\begin{description}
    \item[(a)]
        To show

        \begin{eqnarray*}
            \mbox{{\sc deadlock}} &=& \{(G_1, G_2, \ldots, G_n, P)  :
                \mbox{directed graphs $G_1, G_2, \ldots, G_n$ and the list $P$} \\
                & & \mbox{specify a system with a deadlock state.}\}
            \end{eqnarray*}

        is NP-complete, we must show that it is in NP and that all NP-problems
        are polynomial time reducible to it. To prove the first part, we can
        simply give a certificate and a polynomial time verifier. The
        certificate is a deadlock state $v \in V$.
        The verifier will verify that $v$ is a deadlock state by
        checking if there is a $w \in V$ such that
        $(v, w) \in T$. If there is no $w$ such that $(v,w) \in T$,
        then $v$ is a deadlock state. This process is clearly polynomial
        since the number of states $w \in V$ is polynomially large with respect
        to the input.

        To prove the second part, we show that $\mbox{\sc 3-sat}$ is polynomial
        time reducible to $\mbox{\sc deadlock}$. The following details the
        reduction.

        Our reduction function $f$ will map a 3-CNF formula $\varphi$ to
        $A = (G_1, G_2, \dots, G_n, P)$, where $G_i$ is a directed graph and
        $P$ is a list of pairs of edges, as described in the description of the
        problem. We will map $\varphi$ to $A$ in the following way. For each
        clause, we will make a process/directed graph. The process/directed
        graph will be in the form of a triangle, with each vertex of the
        triangle representing a literal in the clause, and with directed edges
        such that each vertex/state can go to the other to vertices/states.
        So, given a clause $(x_1 \vee x_2 \vee x_3)$, we would generate
        a graph $G_1$ with vertices $v_{11}$ (representing $x_1$), $v_{12}$
        (representing $x_2$), and $v_{13}$ (representing $x_3$), where each
        vertex/state would be able to travel to the other two vertices/state.
        We will then generate list $P$ in the following way. For each literal
        $x_i$, if $\neg x_i$ appears in a separate clause, then we will add
        the pair $((x_i, a), (\neg x_i, b))$, where $a$ is a literal in the same
        clause as $x_i$ and $b$ is a literal in the same clause as $\neg x_i$.
        If there exist multiple clauses with $x_i$ or multiple clauses with
        $\neg x_i$, then we will add in a corresponding pair $((x_i, a), (\neg x_i, b))$
        to $P$ for each additional $x_i$ and $\neg x_i$ pair.

        Now we will show that this reduction function $f$ maps yes to yes and no
        to no. First we will show it maps yes to yes. So, given that $\varphi$
        has a satisfying assignment, we want to show that $A$ has a deadlock
        state. Since $\varphi$ is satisfiable, there is at least one true
        literal in each clause. So the state will be made up of one vertex
        from each graph that corresponds to a true literal in the graph's
        corresponding clause.
        First we will consider the case in which $\varphi$ has all
        $x_i$s and no $\neg x_i$s. Then, our $P$ will be empty, so there will
        be no available transitions, and $A$ will have a deadlock state. Next
        we will consider the case in which $\varphi$ does have $x_i$s and
        $\neg x_i$s, but each $x_i$ and $\neg x_i$ pair is in its own clause.
        Then our $P$ is empty again, so $A$ will have a deadlock state. Finally,
        we will consider the case in which $\varphi$ does have $x_i$ and $\neg
        x_i$ pairs in different clauses/processes. In this case, we also have
        that $A$ has a deadlock state. This is because, since $\varphi$ is a
        satisfiable assignment, only one of $x_i$ and $\neg x_i$ can be true.
        So WLOG, let $x_i$ be true. Then, given how we defined $f$, in one
        process we have the transition $(x_i, a)$ (where $a$ is a literal/vertex
        in the same clause/process as $x_i$) and in another process we have the
        transition $(\neg x_i, b)$ (where $b$ is a literal/vertex
        in the same clause/process as $\neg x_i$). We also have that the pair
        of transitions $((x_i, a), (\neg x_i, b))$ is in $P$. But now we have
        that there is no $w$ such that $(c, w) \in T$, where $c$ is a
        literal/vertex in the same clause/process as $\neg x_i$. This is because
        for $(c, w) \in T$ to be true, $((x_i, a), (c, w))$ must be in $P$. But
        we have that only pairs of the form $((x_i, a), (\neg x_i, b))$ are in
        $P$. This logic can be extended to all the other clauses. In other
        words, the only transitions that can happen are those that involve
        transitioning from $x_i$ to something and $\neg x_i$ to something. But
        the state $v \in V$ (as defined at the top of this section)
        is made up of only one of $x_i$ and $\neg x_i$, not both of them. So
        there can be no transitions.  Thus we have that $A$ has a deadlock state.

        Now we will show that $f$ maps no to no. To prove this, it suffices to
        show that if $A$ is in $\mbox{\sc deadlock}$, then $\varphi$ is
        satisfiable. We have that our $A$ has a deadlock state. So we can simply
        pick all the literals that correspond to the vertices specified by
        the deadlock state. So we have that at least one literal in each clause
        is true, since the state is made up of one vertex from each graph. We
        also have that $x_i$ and $\neg x_i$ cannot both be in the state, since
        if they were, then the state (due to how we defined $f$) would not be a
        deadlock state. So we have that we can construct a satisfiable
        assignment for $\varphi$ and thus we have that no maps to no.

        Now we must only prove that our reduction function $f$ is polynomial
        time computable. For each clause $f$ generates a process/directed graph,
        a process which is clearly polynomial. And for each pair $x_i, \neg
        x_i$ in different clauses, $f$ adds a pair to $P$, a process which is
        also clearly polynomial (bounded by the number of clauses to some power
        $k$). So we have that our reduction function is polynomial.

        Finally, we can conclude that $\mbox{\sc deadlock}$ is NP-complete.

    \item[(b)]
        To prove that $\mbox{\sc reachable deadlock}$ is PSPACE-hard, we
        will reduce from a general $L$ in PSPACE to $\mbox{\sc reachable
        deadlock}$. We will do this by representing $L$ and its corresonding
        TM $M$ as a tableau. We will have the
        rows of our tableau represent the current configuration of our TM $M$ that
        recognizes $L$. These rows will be of polynomial length, since
        $L$ is polynomial space. Our tableau language will be the union of our
        tape language, as well as all the symbols of the form
        $(a, q)$, where $a$ is a tape symbol and $q$ is the state (so the head
        is at the tape symbol $a$ in this case). The first row of our tableau
        will be the initial state of $M$. We will then define our transition
        functions as follows. Basically, the head can either move left, right,
        or stay put. If our initial configuration looks like $a, (b, q), c$,
        then our left transition functions will be of the form
        $(a, q'), b', c$, where $b'$ is in the tape alphabet (possibly different
        than $b$) and $q'$ represents a state (possibly different than $q$).
        The right transition functions will be of the form
        $a, b', (c, q')$, where $b'$ is in the tape alphabet (possibly
        different than $b$) and $q'$ represents a state (possibly different than $q$).
        The stay put transition functions will be of the form
        $a, (b', q'), c$, where $b'$ is in the tape alphabet (possibly
        different than $b$) and $q'$ represents a state (possibly different than $q$).
        For all these transitions, symbols that are not shown are not affected.
        We will also have that the accept state $q_{accept}$ does not transition
        to anything, and that $q_{reject}$ goes to $q_{reject}'$, which then
        goes back to $q_{reject}$, creating an infinite loop (note that we
        will not be adding an infinite number of rows; we will just loop back
        and forth between two rows). Now we will take
        our tableau and reduce it to $\mbox{\sc reachable deadlock}$. So we will
        let each column represent a graph $G_i$. Then we will let $P$ be
        the list of transition functions, modified to fit what $P$ is supposed
        to be (a list of pairs of edges). So for a given transition, we could
        map that to one pair $((e_1, e_1')) \in P$ that is viable given
        the overall state. We will have each cell represent
        the state of a graph (the graph being the column label), so each row
        represents the overall state of all the graphs.

        Now we will show that yes maps to yes and no maps to no. First we will
        show that yes maps to yes. So, given that a string $x$ is in $L$, that means
        our TM $M$ accepts. And since we said that once $q_{accept}$ is reached
        there are no more transitions, this means that we cannot transition
        from one overall state of the graphs to another. So deadlock is reached
        once we reach the accept state.

        Now we will show that no maps to no. So given that a string $x$ is not
        in $L$,
        $M$ rejects. This means $q_{reject}$ is reached, which means we enter
        an infinite loop. So there will always be transitions for the overall
        state of the graphs to take, and deadlock is never reached.

        Now we must only prove that our reduction takes polynomial time and
        space. Our reduction function maps, for each row,
        at most $n^{k}$ cells, where $n$ is
        the input length to $M$, to at most $n^{k}$ states, each of which is
        some constant number of vertices. And there are, in terms of the
        input length $n$, a constant number of rows. So the reduction function
        is polynomial in both space and time.

        Finally we can conclude that $\mbox{\sc reachable deadlock}$ is
        PSPACE-hard.


\end{description}

\newpage

%%%%%%%%%%%%%%%%%%%%%%%%%%%%%% Problem 3
\section*{Problem 3, CS21 Set 7, Matt Lim}
We have that $f(w)$ and $g(w)$ are both computable in space $O(\text{log}n)$.
So to show that $h(w) = f(g(w))$ is computable in $O(\text{log}n)$, we will do
the following. The first tape of our TM $M$ will just be the input $w$. The
third tape of $M$ will be the output $h(w)$. The first part of the second tape
will contain the work needed to compute $g(w)$, which takes up $O(\text{log}n)$
space. Now we must only show that the work needed to compute $f(g(w))$ also
only takes up $O(\text{log}n)$ space. To make this happen, we will compute
$f(g(w))$ bit by bit. That is, when computing $f(g(w))$, we will take one bit
of $g(w)$, work with that, and then output the result onto the third tape. So,
back to the second tape. As we said before, the first part of the second tape
contains the work needed to compute $g(w)$. This is $O(\text{log}n)$ space. The
next part will contain the relevant bit of $g(w)$ that we are working with. We
can have the functionality of telling which bit of $g(w)$ that we need built
into $M$, and we can retrieve the relevant bit using the first part of the
second tape (the work needed to compute $g(w)$.
This is obviously in $O(\text{log}n)$ space. Finally, we will have
the work need to compupte $f(g(w))$, where $g(w)$ is the bit written on the
second part of the tape.
We will now argue that $g(w)$ is in polynomial space. By considering that
$M$ has a finite number of states and a finite sized alphabet, we can see
that the output $g(w)$ has space of order $O(A^{k\text{log}n})$, where $A$ is
the size of the alphabet, $k$ some constant, and $n$ the length of $w$. So
we have that the space of the output $g(w)$ is polynomial. Then we have that,
since $f$ is computable in $O(\text{log}n)$ space, then $f(g(w))$ is computable
in $O(\text{log}n)$ space (since the size of $g(w)$ is polynomially large).
So we have that our second tape is taken up by
3 things that are $O(\text{log}n)$ space, which means that it is
$O(\text{log}n)$ space for each step (for each bit retrieved and used). And
after each step is done, we can just
wipe the last two parts (the bit of $g(w)$ and the work need to compute $f(g(w))$)
and repeat the process until $f(g(w))$ is calculated. Thus we have that
$h(w) = f(g(w))$ is computable in $O(\text{log}n)$ space.
\newpage

%%%%%%%%%%%%%%%%%%%%%%%%%%%%%% Problem 3
\end{document}




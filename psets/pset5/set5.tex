%%%%%%%%%%%%%%%%%%%%%%%%%%%%%% Preamble
\documentclass{article}
\usepackage{amsmath,amssymb,amsthm,fullpage}
\usepackage[a4paper,bindingoffset=0in,left=1in,right=1in,top=1in,
bottom=1in,footskip=0in]{geometry}
\newtheorem*{prop}{Proposition}
%\newcounter{Examplecount}
%\setcounter{Examplecount}{0}
\newenvironment{discussion}{\noindent Discussion.}{}
\pagenumbering{gobble}
\begin{document}

%%%%%%%%%%%%%%%%%%%%%%%%%%%%%% Heading
%\title{CS21 Midterm}
%\author{Matt Lim}
%\date{February 12, 2014}
%\maketitle



%%%%%%%%%%%%%%%%%%%%%%%%%%%%%% Problem 1
\section*{Problem 1, CS21 Set 5, Matt Lim}
\begin{description}
    \item[(a)] False. To prove this we must find a language L that has the
        following properties: L $\in$ EXP and L $\notin$ TIME$(2^{n^{100}}$).
        To find this L, we will use the Time Hierarchy Theorem. Consider
        the proper complexity function $f(n) = 2^{n^{100}} \ge n$. Then we
        have that $f(2n)^{3} = (2^{2n^{100}})^{3}$. So we have that, by
        the Time Hierarchy Theorem,
        TIME($2^{n^{100}}$) $\subsetneq$ TIME($(2^{2n^{100}})^{3}$) $\subseteq$
        EXP. This means that we can choose an L that is in
        TIME($(2^{2n^{100}})^{3}$) (and thus in EXP) that is not in
        TIME($2^{n^{100}}$). This proves (a) to be false.

    \item[(b)] True. This is because we showed in lecture that P $\subsetneq$
        EXP, which means that either P $\neq$ NP or NP $\neq$ EXP (or both).
    \item[(c)] False. Since, due to the problem, we can assume that an
        NP-complete language has an $O(n^{k})$-time algorithm for some fixed
        $k$, we have that NP = P. So then we can do the following.
        To prove this we must find a language L that is in P
        but not in TIME($n^{k}$). To find this L, we will use the Time Hierarchy
        Theorem. Consider the proper complexity function $f(n) = n^{k}$. Then
        we have that $f(2n)^{3} = (2n)^{3k}$. So we have that, by the Time
        Hierarchy Theorem, TIME($n^{k}$) $\subsetneq$ TIME($(2n)^{3k}$)
        $\subseteq$ P. This means that we can choose an L that is in
        TIME($(2n)^{3k}$) (and thus in P) that is not in TIME($n^{k}$). This
        proves (c) to be false.
    \item[(d)] False. Consider the language L = $\Sigma ^{*}$. This language is
        clearly in P (everything is accepted) and thus in NP (by the problem's
        assumption that P = NP). To be NP-complete, every language A in NP
        must reduce to L. But we cannot reduce any language A in NP to L because
        it is impossible to map no to no, since L does not have any instances of
        no to map to. Thus L is not NP-complete, yet is in NP. So (d) is false.
    \item[(e)] True. This is because NP $\subseteq$ EXP. So since every
        EXP-complete language has the characteristic that every language
        A $\in$ EXP reduces to it, and since we have that NP $\subseteq$ EXP and
        thus that every language in NP is in EXP, we can conclude that every
        language in NP reduces to every EXP-complete language.
\end{description}
\newpage

%%%%%%%%%%%%%%%%%%%%%%%%%%%%%% Problem 2
\section*{Problem 2, CS21 Set 5, Matt Lim}
To show
\[ \text{3-COLORABLE} = \{G : G \text{ is 3-colorable}\} \]
is NP-complete, we must show that it is in NP and that all NP-problems are
polynomial time reducible to it.  To prove the first part,
we can simply give a certificate and a polynomial time verifier. The certificate
is just a satisfying coloring of $G$; the verifier will iterate through all
the edges in the graph and check to make sure that no edge has both endpoints
assigned to the same color, and will also check that there are at most 3 colors
used. This can clearly be done in polynomial time.

To prove the second part, we show that 3-SAT is polynomial
time reducible to 3-COLORABLE. The following details the reduction.

Our reduction function $f$ will map a 3-CNF formula $\varphi$ to a graph $G$.
The graph is detailed on the following pages:
\newpage

%%%%%%%%%%%%%%%%%%%%%%%%%%%%%% Problem 3
\section*{Problem 3, CS21 Set 5, Matt Lim}
To show that (3,3)-SAT is NP-complete, we must show that it is in NP and that
all NP-problems are polynomial time reducible to it. To prove the first part,
we can simply give a certificate and a polynomial time verifier. If
$\varphi_{2}$ is a (3,3)-CNF formula, then the certificate is just a
satisfying assignment A for $\varphi_{2}$, and the verifier will check that
this assignment makes every clause in $\varphi_{2}$ true (clearly a polynomial
procedure).  To prove the second
part, we show that 3-SAT is polynomial time reducible to (3,3)-SAT. The
following details the reduction.

Our reduction function $f$ will map a 3-CNF formula $\varphi$ to a new 3-CNF
formula $\varphi_{2}$. If $\varphi$ itself has at most 3 occurrences of any
variable, then we can just make $\varphi_{2}$ a copy of $\varphi$. For this
reduction then clearly yes maps to yes and no maps to no. Else, if
$\varphi$ has more than 3 occurrences of any variable, we can do the following
thing to each of these variables.

Let an arbitrary variable $r$ repeat $n > 3$ times. Then we will introduce
$n$ new variables $y_{1}, y_{2}, \dots , y_{n}$. We want all the $y_{i}$s to
be equivalent to each. That is, we want $y_{1} \Leftrightarrow y_{2}
\Leftrightarrow \dots \Leftrightarrow y_{n}$. To get this, it suffices to show
that $y_{1} \Rightarrow y_{2} \Rightarrow \dots \Rightarrow y_{n} \Rightarrow
y_{1}$. To do this in CNF format, it will look like the following:
\[ (y_{2} \vee \overline{y_{1}}) \wedge (y_{3} \vee \overline{y_{2}})
\wedge \dots (y_{1} \vee \overline{y_{n}}) \]
Then, we will add the above clauses to $\varphi$ (separated by $\wedge$s), and
replace each $i$th instance of $r$ with $y_{i}$.

Now we will show that this reduction function $f$ maps yes to yes and no
to no. First we will show it maps yes to yes. So, given that $\varphi$ has a
satisfying assignment, we want to show that $\varphi_{2}$ is satisfiable
and that it is in (3,3)-SAT. To do this, we will consider our reduction
function. Note that for any arbitrary variable $r$ repeated $n > 3$ times, our
reduction function adds new variables $y_{1}, y_{2}, \dots , y_{n}$, each of
which appear three times (twice in the added clauses (considering both
$y_{i}$ and $\neg {y_{i}}$), once in the replacements). Now
we must consider what happens if $\neg r$ repeats $k > 3$ times as well. That
is, both $r$ and $\neg r$ repeat more than three times. If this is the case,
then we can just add more $y_{i}$s to our equivalence. So we will have
$y_{1}, y_{2}, \dots , y_{n}, y_{n+1}, y_{n+k}$. Then we will replace
each $\neg r_{i}$ with $\neg y_{n+i}$. Note that this still gives us at most
3 occurrences of each variable.
 So, we have that our reduction function gives us a $\varphi_{2}$
that has at most 3 literals per clause and at most 3 occurrences of each
variable. Now we must show that it is satisfiable to complete the proof of
yes maps to yes. To prove this, we must only show that our added clauses
are satisfiable, since our replacement method does not change the
satisfiability of the original clauses (since all the replacements for a
given repeated variable are equivalent). This is clearly true, since we can
just either make all the $y_{i}$'s to be true or all of them to be false.
So we have that yes maps to yes.

Now we will show that this reduction function $f$ maps no to no. To prove this,
consider again the fact that our replacement method does not change the
satisfiability of the original clauses (since all the replacements for a given
repeated variable are equivalent). Thus, the non-added-in clauses in
$\varphi_{2}$ are unsatisfiable. And adding in the extra clauses can do nothing
to change this, since all the clauses are separated by $\wedge$s. So we have
that no maps to no.

Now we must only prove that our reduction function $f$ is polynomial time
computable. A variable can repeat no more than $3n$ times, where $n$ is the number
of clauses. And there can be no more than $3n$ repeated variables. This means
that for each repeated variable, we do no more than $3n$ replacements, and
add no more than $3n + 1$ clauses. Clearly the function is polynomial.
So we have that our reduction function is polynomial.

Finally, we can conclude that (3,3)-SAT is NP-complete.
\newpage

%%%%%%%%%%%%%%%%%%%%%%%%%%%%%% Problem 3
\section*{Problem 4, CS21 Set 5, Matt Lim}
To show that MAX2SAT is NP-complete, we must show that it is in NP and that
all NP-problems are polynomial time reducible to it. To prove the first part,
we can simply give a certificate and a polynomial time verifier. If $\phi$ is
a 2-CNF formula, then the certificate is just an assignment A that
simultaneously satisfies at least $k$ clauses of $\phi$. The verifier will
check that this assignment does indeed satisfy at least $k$ clauses of $\phi$,
a procedure that is clearly polynomial. To prove the second part, we show that
3-SAT is polynomial time reducible to MAX2SAT. The following details the
reduction.

Our reduction function $f$ will map a 3-CNF formula $\varphi$ to a 2-CNF
formula $\phi$. It will do this in the following way. For each clause in
$\varphi$, it will construct 10 new clauses with 2 literals. Let us consider
a single clause $(x, y, z) \in \varphi$. We will map this clause to the
10 new 2-literal clauses shown below:
\[ (x \vee x), (y \vee y), (z \vee z), (w \vee w), \]
\[ (\neg x \vee \neg y), (\neg y \vee \neg z), (\neg z \vee \neg x), \]
\[ (x \vee \neg w), (y \vee \neg w), (z \vee \neg w) \]
Consider the truth table for these 10 clauses, as shown below, letting $1$
represent true, $0$ represent false, and max be the max number of clauses
that are simultaneously satisfiable (depending on what $w$ is assigned):
\begin{center}
  \begin{tabular}{ | l | l | c | r | }
    \hline
    x & y & z & max \\ \hline
    1 & 1 & 1 & 7 \\ \hline
    1 & 0 & 0 & 7 \\ \hline
    0 & 1 & 0 & 7 \\ \hline
    0 & 0 & 1 & 7 \\ \hline
    1 & 1 & 0 & 7 \\ \hline
    0 & 1 & 1 & 7 \\ \hline
    1 & 0 & 1 & 7 \\ \hline
    0 & 0 & 0 & 6 \\ \hline
  \end{tabular}
\end{center}
Note that the maximum number of clauses that can be satisfied here is 7. Also
note that this happens exactly when $(x \vee y \vee z)$ is true. So, this
is our reduction function.

Now we will show that this reduction function $f$ maps yes to yes and no to no.
First we will show yes maps to yes. So, given that $\varphi$ has a satisfying
assignment, we want to show that $\phi$ is a 2-CNF formula for which it is
possible to simultaneously satisfy at least $k$ clauses. So, let $k = 7n$,
where $n$ is the number of clauses in $\varphi$. Since $\varphi$ has a
satisfying assignment, this means that at least one literal in each clause
is true. That is, each arbitrary clause $(x \vee y \vee z) \in \varphi$ is true.
This, given our
reduction function $f$, means that for every 10 2-literal clauses generated
from each 3-literal clause in $\varphi$, 7 of them are true. This then means
that there are $7n$ simultaneously satisfiable clauses in $\phi$. But this
means that at least $k$ clauses can be simultaneously satisfied in $\phi$. So
$\phi$ is in MAX2SAT, and we have that yes maps to yes.

Now we will show that this reduction function $f$ maps no to no. To do this,
it suffices to show that if $\phi$ is in MAX2SAT, then $\varphi$ is satisfiable.
So, consider our $\phi$. More specifically, consider each set of 10 clauses
in $\phi$ that were generated from one clause in $\varphi$. A max of 7 clauses
can be made true from each set. This means that, for $\phi$ to be in MAX2SAT
($k = 7n$ true clauses in this case), 7 clauses in each set of 10 must be true.
But this only happens when each clause that generates each set of 10 2-literal
clauses is true; that is, when each arbitrary $(x \vee y \vee z) \in \varphi$
is true.  In other words, we have that, for every clause in $\varphi$ that
generates each 10 clause set in $\phi$, at least one literal is true. That is,
we have that every clause in $\varphi$ can be made true. And we will never
run into the issue that $x$ and $\neg x$ are both true for some arbitrary $x$,
because this does not occur for an instance of MAX2SAT. So we have that
we can construct a satisfying assignment for $\varphi$, and thus that no maps
to no.

Now we must only prove that our reduction function $f$ is polynomial time
computable. Our function is clearly polynomial, because for each clause
in $\varphi$, it simply generates 10 clauses to put in $\phi$. So we have
that our reduction function is polynomial.

Finally, we can conclude that MAX2SAT is NP-complete.
\end{document}



